\section*{Abstract}

This paper evaluates the methodological advantages of Quantile Regression Forests (QRF) for wealth imputation between the Survey of Consumer Finances (SCF) and Current Population Survey (CPS). We demonstrate that QRF outperforms traditional imputation approaches by preserving conditional distributions rather than merely conditional means, a critical distinction when handling highly skewed wealth data. Our empirical analysis, implemented through the Microimpute package, provides evidence that QRF reduces bias in wealth distribution estimates, achieving a 21-22\% reduction in average quantile loss compared to OLS and hot deck matching. Using 5-fold cross-validation on 22,975 SCF households, QRF maintains an average quantile loss of 0.059 across all quantiles, demonstrating superior distributional accuracy, particularly in the critical 20th-80th percentile range. While we focus on wealth, QRF's advantages extend to any skewed variable requiring distributional preservation, including consumption, medical expenses, and other heavy-tailed economic measures. These technical improvements have substantial implications for wealth inequality research and microdata enhancement. We release our open-source Microimpute package to facilitate microimputation across the field, providing automated method comparison, hyperparameter tuning, and survey weight integration capabilities that streamline the imputation workflow for complex survey data.